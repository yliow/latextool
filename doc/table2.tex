\subsection{\texttt{table2} function}


\begin{python}
s = r'''
from latextool_basic import *
p = Plot()

m = [['','','','','',''],
     ['','','','','',''],
     ['','','','','',''],
     ['','','','','3',''],
     ['','','','','',''],
     ]

table2(p, m, width=0.7, height=0.7, rowlabel='x', collabel='y')
print p
'''.strip()

from latextool_basic import *
execute(s, print_source=True)
\end{python}


\newpage
\subsubsection{Change row and column index values/names}

\begin{python}
s = r"""
from latextool_basic import *
p = Plot()

m = [['','','','','',''],
     ['','','','','',''],
     ['','','','','',''],
     ['','','','','3',''],
     ['','','','','',''],
     ]

table2(p, m, width=0.7, height=0.7, rowlabel='x', collabel='y',
       rownames=[r'\texttt{%s}' % _ for _ in "ABCDE"],
       colnames=[r'\texttt{%s}' % _ for _ in "abcdef"])
print p
""".strip()

from latextool_basic import *
execute(s, print_source=True)
\end{python}


\newpage
\subsubsection{No row and column indices/values}

\begin{python}
s = r"""
from latextool_basic import *
p = Plot()

m = [['','','','','',''],
     ['','','','','',''],
     ['','','','','',''],
     ['','','','','3',''],
     ['','','','','',''],
     ]

table2(p, m, width=0.7, height=0.7, rowlabel='x', collabel='y',
       rownames=[],
       colnames=[])
print p
""".strip()

from latextool_basic import *
execute(s, print_source=True)
\end{python}


\newpage
\subsubsection{Change row/column names}

\begin{python}
s = r"""
from latextool_basic import *

p = Plot()

m = [[0,1,1,0],
     [0,1,0,1],
     [1,0,1,1],
     [0,1,1,1],
     ]

table2(p, m, width=0.7, height=0.7,
       rownames=['00','01','11','10'],
       colnames=['00','01','11','10'],
       rowlabel='$WX$', collabel='$YZ$')      
print p
""".strip()

from latextool_basic import *
execute(s, print_source=True)
\end{python}





\newpage
\subsubsection{No row and no column names}

\begin{python}
s = r"""
from latextool_basic import *
p = Plot()

m = [['','','','','',''],
     ['','','','','',''],
     ['','','','','',''],
     ['','','','','3',''],
     ['','','','','',''],
     ]

table2(p, m, width=0.7, height=0.7,
       rownames=[],
       colnames=[],
       rowlabel=None, collabel=None)
print p
""".strip()

from latextool_basic import *
execute(s, print_source=True)
\end{python}



\newpage\subsubsection{\texttt{border\_linewidth}}

\begin{python}
s = r"""
from latextool_basic import *
p = Plot()

m = [['','','','','',''],
     ['','','','','',''],
     ['','','','','',''],
     ['','','','','3',''],
     ['','','','','',''],
     ]

table2(p, m, x=0, y=0, width=0.7, height=0.7,
       rownames=[],
       colnames=[],
       rowlabel=None, collabel=None, border_linewidth=0)
table2(p, m, x=8, y=0, width=0.7, height=0.7,
       rownames=[],
       colnames=[],
       rowlabel=None, collabel=None, border_linewidth=0.5)
print p
""".strip()

from latextool_basic import *
execute(s, print_source=True)
\end{python}





\newpage
\subsubsection{Adding things beneath table2: background}

The \verb!do_not_plot! can be used to compute the coordinate of the cells.
\verb!table2! will then return the RectContainer object but the table is not drawn.
So one way to change the background is to do \verb!do_not_plot!
to get the coordinates, draw a rect with background color,
then call \verb!table2! to draw the contents of the table.
(Also, see rect parameter later.)
\begin{python}
s = r'''
from latextool_basic import *

m = [['','','','','',''],
     ['','','','','',''],
     ['','','','','',''],
     ['','','','','3',''],
     ['0','',''],           # OK to have jagged arrays
     ]

p = Plot()
C = table2(p, m, rowlabel='x', collabel='y', do_not_plot=True)

# Draw over C[1][4]
x0,y0 = C[1][4].bottomleft()
x1,y1 = C[1][4].topright()
p += Rect(x0=x0, y0=y0, x1=x1, y1=y1,
          background='blue!20',label='A', linewidth=0)

# Draw table one more time to get border of C[1][4] correct
table2(p, m, rowlabel='x', collabel='y')

p += Line(points=[C[0][1].center(), C[2][3].center()], endstyle='>')

print p
'''.strip()

from latextool_basic import *
execute(s, print_source=True)
\end{python}









\newpage
\subsubsection{Using cells}

Drawing arrows between cells using \texttt{shorten} function for arrows
using a shortening factor:

\begin{python}
s = r"""
x = "     "
y = "      "
m = [['' for i in range(len(y))] for j in range(len(x))]
m[3][2] = 15
m[2][1] = 26
from latextool_basic import *
p = Plot()
C = table2(p, m, width=0.7, height=0.7,
           rowlabel=r'\texttt{x}', collabel=r'\texttt{y}')
           
p0 = C[3][2].center()
p1 = C[2][1].center()
p0, p1 = shorten(p0, p1, factor=0.5)
p += Line(points=[p0, p1], endstyle='>')

p0 = C[3][2].center()
p1 = C[2][2].center()
p0, p1 = shorten(p0, p1, factor=0.5)
p += Line(points=[p0, p1], endstyle='>')

p0 = C[3][2].center()
p1 = C[3][1].center()
p0, p1 = shorten(p0, p1, factor=0.5)
p += Line(points=[p0, p1], endstyle='>')

print p
""".strip()

from latextool_basic import *
exec(s)
print console(s)
\end{python}





\newpage
Drawing arrows between cells using \texttt{shorten} function for arrows
using a shortening amount:

\begin{python}
s = r"""
x = "     "
y = "      "
m = [['' for i in range(len(y))] for j in range(len(x))]
from latextool_basic import *
p = Plot()
C = table2(p, m, width=0.7, height=0.7,
           rowlabel=r'\texttt{x}', collabel=r'\texttt{y}')
           
p0 = C[1][1].center()
p1 = C[3][1].center()
p += Line(points=[p0, p1], endstyle='>', linecolor='red')

p0 = C[1][2].center()
p1 = C[3][2].center()
p0, p1 = shorten(p0, p1, start_by=0.5)
p += Line(points=[p0, p1], endstyle='>', linecolor='green')

p0 = C[1][3].center()
p1 = C[3][3].center()
p0, p1 = shorten(p0, p1, end_by=0.5)
p += Line(points=[p0, p1], endstyle='>', linecolor='blue')

print p
""".strip()

from latextool_basic import *
exec(s)
print console(s)
\end{python}





\newpage
\subsubsection{Rect for cells}

You can specify a rect function of your own:
\begin{python}
s = r"""
from latextool_basic import *

c = RectContainer(x=0, y=0)
xs = 'ABA'; shadedbackground='black!10'
for i,x in enumerate(xs):
    if i in [1,2]:
        c += Rect2(x0=0, y0=0, x1=0.4, y1=0.4,
             background=shadedbackground,
             linewidth=0.01, label=r'{\texttt{%s}}' % x)
    else:
        c += Rect2(x0=0, y0=0, x1=0.4, y1=0.4,
                   background='white',
                   linewidth=0.01, label=r'{\texttt{%s}}' % x)
q = Plot(); q += c
m = [[str(q),'','','','',''],
     ['','','','','',''],
     ['','','','','3',''],
     ]

p = Plot()

def rect(x):
    return Rect(x0=0, y0=0, x1=2, y1=2, radius=0.2, innersep=0.2,
    linecolor='red!30!white', linewidth=0.1,
    background='blue!20!white!', foreground='blue!90!white',
    s='%s' % x, align='t')

table2(p, m, width=0.7, height=0.7, rowlabel='x', collabel='y', rect=rect)
print p
""".strip()

from latextool_basic import *
exec(s)
print r'''
{\scriptsize
%s
}
''' % console(s)
\end{python}




\newpage
\subsubsection{Using rect to change the background}

You can use the rect to for instance change
the background.
In the following, the rects at column 2
are given larger width and a different background,
and the contents are placed in a minipage with tiny fontsize.

\begin{python}
s = r'''
from latextool_basic import *
m = [['1','primes: 2, 3, 5, 7, 11, 13, 19, ...','','','',''],
     ['','','','','',''],
     ['','','','','',''],
     ['','','','','3',''],
     ]
def get_rect():
    i = [0]
    def rect(x):
        if i[0] % 6 == 1:
            i[0] += 1
            return Rect(x0=0, y0=0, x1=2, y1=1,
                        background='blue!20!white!',
                        innersep=0.1,
                        s=r'{\tiny %s}' % x, align='t')
        else:
            i[0] += 1                
            return Rect(x0=0, y0=0, x1=1, y1=1, label=x)
    return rect
p = Plot()
table2(p, m, rowlabel='x', collabel='y', rect=get_rect())
print p
'''.strip()

from latextool_basic import *
execute(s, print_source=True)
\end{python}
