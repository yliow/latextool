\subsection{UML: sequence diagram}

\begin{python}
s = r'''
from latextool_basic import *
p = Plot()

klsnames = ['main:Main', ':DiceGame', 'die1:Die', 'die2:Die']

messages = [(':DiceGame','play()'),
            ('die1:Die', 'roll()'),
            ('return',''),
            ('die1:Die', 'getFaceValue()'),
            ('return','faceValue'),
            ('die2:Die', 'roll()'),
            ('return',''),
            ('die2:Die', 'getFaceValue()'),
            (None,''), # no return line
            ('return',''),
            (None,''), 
           ]

X,Y = 0,0
dY = -1
width = 2.5
hsep = 1.5
worldlinelength = 5
activationbarwidth = 0.4

def kls(p, x, y, s):
    return uml_class(p, x, y, width=width, classname=s, showempty=False)

k = {}
x = X
for klsname in klsnames:
    k[klsname] = kls(p, x, Y, klsname)
    x += width + hsep
    
def activationbar(klsname, y0, y1):
    p0 = k[klsname].bottom()
    x0 = p0[0] - activationbarwidth / 2.0
    x1 = p0[0] + activationbarwidth / 2.0 
    return Rect(x0=x0, y0=y0, x1=x1, y1=y1, background='white')

Y = k.values()[0].bottom()[1] # start y coordinate at the bottom of the class
                              # boxes
Y += dY

f = 'main:Main'
stack = [(f, Y)]
activationbars = [] # to be drawn AFTER the timeline
display_return = True

for x in messages:
    if x[0] == 'return':
        Y += dY
        s = x[1]
        g, Y1 = stack.pop()
        if s != '' or display_return:
            a, b = uml_functioncall(k, klsname0=f, klsname1=g, y=Y, s=s,
                   activationbarwidth=activationbarwidth,
                   linestyle='dashed')
        p += a; p += b
        activationbars.append((f, Y, Y1))
        f = g
    elif x[0] == None:
        Y += dY
        s = x[1]
        g, Y1 = stack.pop()
        #if s != '' or display_return:
        #    a, b = uml_functioncall(k, klsname0=f, klsname1=g, y=Y, s=s,
        #           activationbarwidth=activationbarwidth,
        #           linestyle='dashed')
        #p += a; p += b
        activationbars.append((f, Y, Y1))
        f = g
    else:
        Y += dY
        g, s = x
        a, b = uml_functioncall(k, klsname0=f, klsname1=g, y=Y, s=s,
               activationbarwidth=activationbarwidth)
        p += a; p += b
        stack.append((f, Y))
        f = g

Y -= 0.5
# timelines
for klsname in ['main:Main', ':DiceGame', 'die1:Die', 'die2:Die']:
    p0 = k[klsname].bottom()
    p1 = (p0[0], Y)
    p += Line(points=[p0,p1], linestyle='dashed')

# activation bars must be drawn after worldline
for a, b, c in activationbars:
    p += activationbar(a, b, c)
    
#p += Grid()

print(p)
'''.strip()

from latextool_basic import *
print('{\scriptsize %}' % console(s))
\end{python}


\begin{python}
s = r'''
from latextool_basic import *
p = Plot()

klsnames = ['main:Main', ':DiceGame', 'die1:Die', 'die2:Die']

messages = [(':DiceGame','play()'),
            ('die1:Die', 'roll()'),
            ('return',''),
            ('die1:Die', 'getFaceValue()'),
            ('return','faceValue'),
            ('die2:Die', 'roll()'),
            ('return',''),
            ('die2:Die', 'getFaceValue()'),
            (None,''), # no return line
            ('return',''),
            (None,''), 
           ]

X,Y = 0,0
dY = -1
width = 2.5
hsep = 1.5
worldlinelength = 5
activationbarwidth = 0.4

def kls(p, x, y, s):
    return uml_class(p, x, y, width=width, classname=s, showempty=False)

k = {}
x = X
for klsname in klsnames:
    k[klsname] = kls(p, x, Y, klsname)
    x += width + hsep
    
def activationbar(klsname, y0, y1):
    p0 = k[klsname].bottom()
    x0 = p0[0] - activationbarwidth / 2.0
    x1 = p0[0] + activationbarwidth / 2.0 
    return Rect(x0=x0, y0=y0, x1=x1, y1=y1, background='white')

Y = k.values()[0].bottom()[1] # start y coordinate at the bottom of the class
                              # boxes
Y += dY

f = 'main:Main'
stack = [(f, Y)]
activationbars = [] # to be drawn AFTER the timeline
display_return = True

for x in messages:
    if x[0] == 'return':
        Y += dY
        s = x[1]
        g, Y1 = stack.pop()
        if s != '' or display_return:
            a, b = uml_functioncall(k, klsname0=f, klsname1=g, y=Y, s=s,
                   activationbarwidth=activationbarwidth,
                   linestyle='dashed')
        p += a; p += b
        activationbars.append((f, Y, Y1))
        f = g
    elif x[0] == None:
        Y += dY
        s = x[1]
        g, Y1 = stack.pop()
        #if s != '' or display_return:
        #    a, b = uml_functioncall(k, klsname0=f, klsname1=g, y=Y, s=s,
        #           activationbarwidth=activationbarwidth,
        #           linestyle='dashed')
        #p += a; p += b
        activationbars.append((f, Y, Y1))
        f = g
    else:
        Y += dY
        g, s = x
        a, b = uml_functioncall(k, klsname0=f, klsname1=g, y=Y, s=s,
               activationbarwidth=activationbarwidth)
        p += a; p += b
        stack.append((f, Y))
        f = g

Y -= 0.5
# timelines
for klsname in ['main:Main', ':DiceGame', 'die1:Die', 'die2:Die']:
    p0 = k[klsname].bottom()
    p1 = (p0[0], Y)
    p += Line(points=[p0,p1], linestyle='dashed')

# activation bars must be drawn after worldline
for a, b, c in activationbars:
    p += activationbar(a, b, c)
    
#p += Grid()

print(p)
'''.strip()

from latextool_basic import *
execute(s)
\end{python}
